\documentclass{article}

\addtolength{\hoffset}{-2cm}
\addtolength{\textwidth}{3.5cm}
\addtolength{\voffset}{-2cm}
\addtolength{\textheight}{3cm}

\usepackage{amsmath}		 	
\usepackage{amsfonts}	 	
\usepackage{amssymb}
\usepackage{graphicx}	
\usepackage{fancyhdr}	 		
\usepackage{hyperref}
\usepackage{pdflscape}
\thispagestyle{fancy}
\chead{Kalman Filter}
\rhead{}

\begin{document}
\section{Linear Kalman Filter}
\begin{itemize}
\item System Model: $$x_{t} = Fx_{t-1} + Gu_t + w_t$$
\begin{itemize}
\item F is the transition matrix
\item $x_{t-1}$ is the state vector
\item G is a transition matrix, $u_t$ is a input (control) vector
\item $w_t$ is process noise vector, $w_t \sim$ N(0, $Q$)
\end{itemize}

\item Observation Model: $$z_{t} = Hx_{t} + v_t$$
\begin{itemize}
\item H is a transition matrix, $x_{t}$ is the state vector
\item $v_{t}$ is the observation noise vector, $v_t \sim$ N(0, $R$)
\end{itemize}

\item Assumptions
\begin{itemize}
\item $w$ and $v$ are uncorrelated
\item Initial system state, $x_0$ is uncorrelated to $w$ and $v$
\item The initial conditions, $\hat x _{0|0} = E(x_0), P_{0|0}=E((\hat x _{0|0}- x_0)(\hat x _{0|0}-x_0)^T)$ are known
\end{itemize}
\item Notation Note: $\hat x_{t|i}$ is estimation of $x$ at time $t$ based on all the observations up to and including time $i$
\end{itemize}

\begin{itemize}
\item First, we would like an estimate of our state vector $x_t$, using $z_1, z_2, ... z_{t-1}$, our observations up to and including time $t-1$:
\begin{itemize}
\item Predicted state estimate: $$\hat x_{t|t-1} = F\hat x_{t-1|t-1} + Gu_{t-1}$$   
\item Predicted error covariance matrix: $$P_{t|t-1} = F P_{t-1|t-1}F^T + Q$$
\end{itemize}

\item Then at time $t$ we observe a new observation, $z_t$. We would like to use this observation to update our original state estimate $\hat x_{t|t-1}$:

\begin{itemize}
\item Update the state estimate: $$\hat x_{t|t} = (I - K_tH)\hat x_{t|t-1} + K_t z_t $$ 
\item Update the error covariance: $$P_{t|t} = (I - K_tH ) P_{t|t-1}$$
\item Where the Kalman Gain, $K_t$, is given by $$K_t = P_{t|t-1}H^T [H P_{t|t-1} H^T + R]^{-1}$$
\end{itemize}

\item Other Notation
\begin{itemize}
\item Innovation Gain at time $k$: $$ z_t =  H\hat x_{t|t-1}$$
\item Innovation Covariance matrix at time $k$: $$S = H P_{t|t-1} H^T + R $$
\end{itemize}

\end{itemize}
%----------------------------Extended Kalman Filter---------------------------------------------
\newpage
\section{Extended Kalman Filter}
 \begin{itemize}
\item System Model: $$x_{t} = f(x_{t-1},  u_t)+ w_t$$
\begin{itemize}
\item f is an arbitrary nonlinear function 
\item $u_t$ is a input (control) vector
\item $w_t$ is process noise vector, $w_t \sim$ N(0, $Q_t$)
\end{itemize}

\item Observation Model: $$z_{t} = h(x_{t}) + v_t$$
\begin{itemize}
\item h is an arbitrary nonlinear function, $x_{t}$ is the state vector
\item $v_{t}$ is the observation noise vector, $v_t \sim$ N(0, $R_t$)
\item for covariance estimation, the nonlinear terms in system and observation model are linearized using first order Taylor series approximation
\end{itemize}
\item Prediction Equations (Priori):
\begin{itemize}
\item State equation  $$\hat x_{t|t-1}=f(\hat x_{t-1|t-1},u_{t})$$
\item Covariance equation $$P_{t|t-1} = F_{t-1} P_{t-1|t-1}F_{t-1}^T + Q_t$$
\item where $F_{t-1}$ is $$F_{t-1} =\left. \frac{\partial f}{\partial x} \right |_{\hat x_{t-1|t-1},u_t}$$
\end{itemize}
\item Update Equations (Posteriori):
\begin{itemize}
\item Kalman Gain $$K_t=P_{t|t-1}H_t^T \left [ H_tP_{t|t-1}H^T+R_t\right]^{-1}$$
\item State equation $$\hat x_{t|t}=\hat x_{t|t-1} + K_t\left( z_t-h(X_{t|t-1}\right)$$
\item Covariance $$P_{t|t} = \left( 1-K_tH_t\right ) P_{t|t-1}$$
\item where $H_t$ is $$H_t =\left. \frac{\partial h}{\partial x} \right |_{\hat x_{t|t-1}}$$
\end{itemize}
\item EKF is not optimal because of the error due to linear approximation of nonlinearity
\subsection{Non-Additive Noise}
\item Model Equation $$x_{t} = f(x_{t-1},  u_t, w_t)$$    $$z_{t} = h(x_{t}, v_t)$$
\begin{itemize}
\item Where $w_t \sim$ N(0, $\tilde Q_t$) and  $v_t \sim$ N(0, $\tilde R_t$)
\end{itemize}
\item Noise terms can also be linearized using first order taylor series approximation and the additive noise covariance terms in EKF equations become 
$$Q_t=L_t\tilde Q_t L_t^T$$ $$R_t = M_t\tilde R_t M_t^T$$
\begin{itemize}
\item where L and M are $$L_t=\left. \frac{\partial f}{\partial w} \right |_{w_t}$$ $$M_t=\left. \frac{\partial f}{\partial v} \right |_{v_t}$$
\end{itemize}

\section{Parameter Estimation}
\item Model: $$y=G(x,w)$$
\begin{itemize}
\item where $x$ is the input of the process, $y$ is the output of the process and $G$ is the nonlinear map parametrized by the vector $w$. 
\item let \{$x_k$, $\tilde y_k$ \} k=1,2..,n are the pairs consisting of inputs and measured outputs. 
\item The error is $e_k=\tilde y_k - G(x_k,w)$
\end{itemize}
\item State space model can be rewritten to estimate the parameter $w$ 
$$w_k=w_{k-1}+u_k$$ $$y_k=G(x_k,w_k)+e_k$$
\begin{itemize}
\item In the context of EKF $w$ is treated as stationary process driven by noise $u$  and $y$ is the nonlinear observation
\item $\hat w=E[w|y]$ is the optimal value of the parameter $w$, which can be recursively estimated using EKF in section 2.
\item Convergence depends on the choice of $u_k$
\end{itemize} 
\item A similar framework can be used for dual estimation of state and parameter (see reference). 

\end{itemize}


\begin{landscape}

\section{Kalman Filtering in R}
\vspace{5pt}

\begin{tabular}{c | c | c | c | c}		
 & \verb|FKF| & \verb|dlm| & \verb|dse| & \verb|KFAS| \\ 
 \hline
 System& $\alpha_{t} = T\alpha_{t-1} + d + H\eta_t$ 
 & $\alpha_{t} = G\alpha_{t-1} + w_t$ & $\alpha_{t} = F\alpha_{t-1} + Gu_t + Q\eta_t$& $\alpha_{t} = T\alpha_{t-1}+ R\eta_t$\\
 
Model & $\eta_t \sim N(0, I) $ & $  w_t \sim N(0, W)$ &$ \eta_t \sim N(0, I) $ & $\eta_t \sim N(0, Q)$\\
& $H\eta_t \sim N(0, HH^T)$ & &$ Q\eta_t \sim N(0, QQ^T)$ & $R\eta_t \sim N(0, RQR^T)$\\
\hline  

 Observation & $y_{t} = Z\alpha_{t} + c + G\epsilon_t$
 & $y_{t} = F\alpha_{t} + v_t$ &$z_{t} = H\alpha_{t} + R\epsilon_t$ &$ y_{t} = Z\alpha_{t} + \epsilon_t$\\
 
Model & $\epsilon_t \sim N(0,I) $ & $v_t \sim N(0, V)$& $\epsilon_t \sim N(0,I)$& $\epsilon_t \sim N(0,H)$\\
& $G\epsilon_t \sim N(0, GG^T)$ & & $R\epsilon_t \sim N(0, RR^T)$& \\
\hline  
Function & \verb|fkf(a0, P0, dt, ct, Tt,|& \verb|dlm(m0, C0, FF, V, GG, W)|& \verb|SS (F, G, H, Q, R, z0, P0)|& \verb|KFAS(a1, P1, Z, T,| \\

& \verb|Zt, HHt, GGt, yt)|& & & \verb|H, Q, R, y, u)|\\
\hline
Function & \verb|a0| = initial mean & \verb|m0| = initial mean& \verb|z0| = initial mean& \verb|a1| = initial mean\\

Arguments& \verb|P0| = initial covariance & \verb|C0| = initial covariance & \verb|P0| = initial covariance& \verb|P1| = initial covariance\\

& \verb|dt, ct| = $d, c$ & \verb|FF, GG| = $F, G$ & \verb|F, G, H| = $F, G, H$& \verb|T, Z| = $T, Z$\\

& \verb|Tt, Zt| = $Z, T$ & \verb|V, W| = $V, W$& \verb|Q, R| = $Q, R$& \verb|H, Q, R| = $H, Q, R$\\

& \verb|HHt, GGt| = $HH^T, GG^T$& & &\verb|y| = data\\
& & & &\verb|u| = parameters in Non-Gaussian case\\
\hline
\hline

Non-Gaussian &  &  &  & $\checkmark$ \\ 
\hline
Time Variant & $\checkmark$& $\checkmark$&  & $\checkmark$\\ 
\hline
Missing Values & $\checkmark$ & $\checkmark$&   & $\checkmark$ \\ 
\hline
\hline
\end{tabular}

\vspace{30pt}
\end{landscape}

%**********************************************************
\section{Exercises}

\subsection{\tt dlm}
\subsubsection*{Required File}
\verb|dlmlab.rmd| 
\subsubsection*{Exercise}
Go through the \verb|dlmlab.r| file to understand how the \verb|dlm| package works with a simple linear model. To familiarize yourself with the different \verb|R| packages one can use for Kalman Filtering, try repeating this example with one of the following packages: \verb|FKF, dse, KFAS| 


\subsection{EKF - Tracking}
\subsubsection*{Required File}
 \verb|trackinglab.r|
\subsubsection*{Background Information}
A logistic growth model with rate $r$ and carrying capacity $k$ can be written as
$$\frac{dp}{dt}=rp\left( 1-\frac{p}{k}\right)$$
with initial guess $p_0$, the logistic growth model can be analytically solved as.
$$p=\frac{kp_0exp(rt)}{k+p_0(exp(rt)-1)}$$
A population data is modeled based on the analytical solution of logistic growth model with additive random process error
$$p_{t}=\frac{kp_{t-1}exp(r \Delta t)}{k+p_{t-1}(exp(r \Delta t)-1)} + v$$
An extended Kalman Filter is used to track the population given that the variance of process and observation error is known. state space is assumed as $x=[r \text{  }  p]^T$ and observation model is given as
$$z=[0\text{  }  1][r\text{  }  p]^T+w$$
In the given code the sample data is  synthetically generated and tracking algorithm is implemented.

\subsubsection*{Exercise 1}
In the sample code only population in the state space is assumed to have additive process error. Modify the code such that rate of population growth also has additive process error. 
\subsubsection*{Exercise 2}
Implement Kalman filter tracking algorithm for the same logistic population growth by using Euler's explicit time stepping scheme.
$$p_t=p_{t-1}+\Delta t \left( rp_{t-1}\left( 1-\frac{p_{t-1}}{k}\right)+ v\right )$$

\subsection{EKF - Parameter Estimation}
\subsubsection*{Required File}
\verb|paramestimationlab.r|
\subsubsection*{Background Information}
A sample code for estimating parameter a in the equation $$y=a^2x^2+x+1$$ is given in the sample code.  The data is synthetically generated to observe the error of estimated parameter form the true value of the parameter.
\subsubsection*{Exercise 1}
Estimate the initial velocity ($v_0$) of the projectile for the given data in projectile.txt using initial angle as $\theta = pi/4$. Projectile motion equation is 
$$x=v_0t\cos(\theta)$$
$$y=v_0t\sin(\theta)-\frac{gt^2}{2}$$
\section{References}

Linear Kalman Filter
\begin{itemize}
\item \href{http://www.cs.unc.edu/~tracker/media/pdf/SIGGRAPH2001_CoursePack_08.pdf}{An Introduction to the Kalman Filter}
\item \href{http://www.robots.ox.ac.uk/~ian/Teaching/Estimation/LectureNotes2.pdf}{Derivation of Kalman Filter}
\item \href{https://en.wikipedia.org/wiki/Kalman_filter}{Wikipedia Article on Kalman Filter}
\end{itemize}
Extended Kalman Filter
\begin{itemize}
\item Yaakov Bar-Shalom, X. Rong Li, Thiagalingam Kirubarajan; \textbf{Estimation with Applications to Tracking and Navigation}; Wiley Interscience Publication.
\item \href{https://en.wikipedia.org/wiki/Extended_Kalman_filter}{Wikipedia Article on Extended Kalman Filter}
\end{itemize}
Parameter Estimation
\begin{itemize}
\item Simon Haykin; \textbf{Kalman Filtering and Neural Networks}; Wiley Publications.
\item Wan E. A.; The unscented Kalman filter for nonlinear estimation; Adaptive Systems for Signal Processing, Communications, and Control Symposium 2000. AS-SPCC. The IEEE 2000.
\end{itemize}
\verb|R| Resources
\begin{itemize}
\item \href{http://core.ac.uk/download/pdf/6340262.pdf}{Kalman Filtering in R}
\item \href{https://cran.r-project.org/web/packages/FKF/FKF.pdf}{\tt FKF}
\item \href{https://cran.r-project.org/web/packages/dlm/dlm.pdf}{\tt dlm}
\item \href{https://cran.r-project.org/web/packages/dse/dse.pdf}{\tt dse}
\item \href{https://cran.r-project.org/web/packages/KFAS/KFAS.pdf}{\tt KFAS}
\end{itemize}
\end{document}
