\documentclass{article}

\addtolength{\hoffset}{-2cm}
\addtolength{\textwidth}{3.5cm}
\addtolength{\voffset}{-2cm}
\addtolength{\textheight}{3cm}

\usepackage{amsmath}		 	
\usepackage{amsfonts}	 	
\usepackage{graphicx}			
\usepackage{fancyhdr}	 		
\usepackage{hyperref}
\pagestyle{fancy}
\chead{Extended Kalman Filter - Lab}
\rhead{}

\begin{document}
\section{EKF - Tracking}
A logistic growth model with rate $r$ and carrying capacity $k$ can be written as
$$
\frac{dp}{dt}=rp\left( 1-\frac{p}{k}\right)
$$
with initial guess $p_0$, the logistic growth model can be analytically solved as.
$$
p=\frac{kp_0\exp(rt)}{k+p_0(\exp(rt)-1)}
$$
Population data is modeled based on the analytical solution of the logistic growth model with additive random process error
$$
p_{t}=\frac{kp_{t-1}\exp(r \Delta t)}{k+p_{t-1}(\exp(r \Delta t)-1)} + v
$$

% use \> for spacing:
% http://tex.stackexchange.com/questions/74353/what-commands-are-there-for-horizontal-spacing
An extended Kalman Filter is used to track the population given that the variance of process and observation error is known. The state space is assumed to be $x=[r \>  p]^T$ and observation model is given as
$$
z=[0 \>  1][r \>  p]^T+w
$$
In the given code the sample data is  synthetically generated and tracking algorithm is implemented.

\subsection{Exercise 1}
In the sample code only the population size is assumed to have additive process error. Modify the code such that rate of population growth also has additive process error. 

\subsection{Exercise 2}
Implement the Kalman filter tracking algorithm for the same logistic population growth by using Euler's explicit time stepping scheme.
$$
p_t=p_{t-1}+\Delta t \left( rp_{t-1}\left( 1-\frac{p_{t-1}}{k}\right)+ v\right )
$$

\section{EKF - Parameter Estimation}

A sample code for estimating parameter $a$ in the equation $y=a^2x^2+x+1$ is given in the sample code.  The data is synthetically generated to observe the error of estimated parameter from the true value of the parameter.

\subsection{Exercise 3}
Estimate the initial velocity ($v_0$) of the projectile for the given data in {\tt projectile.txt} using the initial angle as $\theta = \pi/4$. 
The equation of the projectile's motion is

\begin{equation*}
\begin{split}
x & =v_0t\cos(\theta) \\
y & =v_0t\sin(\theta)-\frac{gt^2}{2}
\end{split}
\end{equation*}

\end{document}
